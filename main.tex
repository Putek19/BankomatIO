\documentclass[a4paper,12pt]{article}

% --- Pakiety ---
\usepackage[utf8]{inputenc}
\usepackage[T1]{fontenc}
\usepackage[polish]{babel}
\usepackage{graphicx}       % Do wstawiania diagramów
\usepackage{geometry}       % Marginesy
\usepackage{listings}       % Do kodu źródłowego
\usepackage{xcolor}         % Kolory
\usepackage{float}          % Pozycjonowanie obrazków

% --- Marginesy ---
\geometry{
 a4paper,
 total={170mm,257mm},
 left=20mm,
 top=20mm,
}

% --- Konfiguracja wyglądu kodu Java ---
\definecolor{javared}{rgb}{0.6,0,0} 
\definecolor{javagreen}{rgb}{0.25,0.5,0.35}
\definecolor{javapurple}{rgb}{0.5,0,0.35}
\definecolor{javadocblue}{rgb}{0.25,0.35,0.75}

\lstset{
    language=Java,
    basicstyle=\ttfamily\footnotesize,
    keywordstyle=\color{javapurple}\bfseries,
    stringstyle=\color{javared},
    commentstyle=\color{javagreen},
    morecomment=[s][\color{javadocblue}]{/**}{*/},
    numbers=left,
    numberstyle=\tiny\color{black},
    stepnumber=1,
    numbersep=10pt,
    tabsize=4,
    showspaces=false,
    showstringspaces=false,
    breaklines=true,
    frame=single,
    captionpos=t, % Tytuł nad ramką
    inputencoding=utf8,
    extendedchars=true,
    literate={ą}{{\k{a}}}1 {ć}{{\'c}}1 {ę}{{\k{e}}}1 {ł}{{\l{}}}1 {ń}{{\'n}}1 {ó}{{\'o}}1 {ś}{{\'s}}1 {ż}{{\.z}}1 {ź}{{\'z}}1 {Ą}{{\k{A}}}1 {Ć}{{\'C}}1 {Ę}{{\k{E}}}1 {Ł}{{\L{}}}1 {Ń}{{\'N}}1 {Ó}{{\'O}}1 {Ś}{{\'S}}1 {Ż}{{\.Z}}1 {Ź}{{\'Z}}1
}


\title{\textbf{Sprawozdanie}\\Inżynieria Oprogramowania - Laboratoria 6-8}
\author{
    \textbf{Autorzy:} \\
    Jakub Nowacki, 281137 \\
    Daniel Abbas, 281134
}
\date{\today}

\begin{document}

\maketitle

% --- Wstęp ---
\begin{flushleft}
    \textbf{Nazwa systemu:} System dla bankomatu. \\
    \textbf{Temat:} Modelowanie obiektowej 3-warstwowej architektury.
\end{flushleft}

\tableofcontents
\newpage

% =========================================================
% ZADANIA 1-3: DIAGRAMY
% =========================================================

\section{Zadanie 1: Diagram Komponentów}
Poniższy diagram przedstawia architekturę systemu podzieloną na warstwy: Widok, Kontroler i Model.

\begin{figure}[H]
    \centering
    % Odkomentuj poniższą linię i wstaw nazwę swojego pliku PNG
    \includegraphics[width=1.0\textwidth]{komponent.png}
    
    % To jest ramka zastępcza - usuń ją po wstawieniu obrazka
    
    
    \caption{Diagram komponentów systemu bankomatu}
\end{figure}

\newpage

\section{Zadanie 2: Diagram Klas Komponentu Kontroler}
Diagram przedstawia strukturę warstwy kontroli z wykorzystaniem wzorców Fasada i Strategia.

\begin{figure}[H]
    \centering
    % Odkomentuj poniższą linię i wstaw nazwę swojego pliku PNG
    \includegraphics[width=1.0\textwidth]{kontroler.png}
    
    % To jest ramka zastępcza - usuń ją po wstawieniu obrazka
    
    
    \caption{Diagram klas komponentu Kontroler}
\end{figure}

\newpage

\section{Zadanie 3: Diagram Klas Komponentu Model}
Diagram przedstawia strukturę warstwy modelu z wykorzystaniem wzorców: Adapter (DAO), Dekorator (Karta) i Metoda Wytwórcza.

\begin{figure}[H]
    \centering
    % Odkomentuj poniższą linię i wstaw nazwę swojego pliku PNG
    \includegraphics[width=1.0\textwidth]{model.png}
    
    % To jest ramka zastępcza - usuń ją po wstawieniu obrazka
    
    
    \caption{Diagram klas komponentu Model}
\end{figure}

\newpage

% =========================================================
% ZADANIE 4: KOD ŹRÓDŁOWY
% =========================================================

\section{Zadanie 4: Implementacja kodu}
Poniżej przedstawiono implementację klas wygenerowaną na podstawie diagramów i uzupełnioną o logikę symulacji.

% --- PAKIET KONTROLER ---
\subsection{Pakiet Kontroler}

\begin{lstlisting}[caption={Kontroler/SystemBankomatu.java (Main)}]
package Kontroler;

import Model.*;

public class SystemBankomatu {

	public static void main(String[] Args) {
		// Inicjalizacja systemu
		DAO dao = new DAO();
		Inwentarz inwentarz = new Inwentarz(dao);
		Model model = new Model(inwentarz, dao);
		
		// Utworzenie przykładowego klienta
		Formularz formularz = new Formularz("Jan", "Kowalski", 123456789);
		FabrykaKlienta fabryka = new FabrykaKlienta();
		IKlient klient = fabryka.stworzKontoKlienta(formularz);
		inwentarz.dodajKlienta(klient);
		inwentarz.dajKlienta(1);
		
		// Utworzenie przykładowej karty
		Karta karta = new Karta(1, "1234", new java.math.BigDecimal("1000.00"));
		klient.dodajKarte(karta);
		
		// Inicjalizacja kontrolerów
		KontrolerKlienta kontrolerKlienta = new KontrolerKlienta(model);
		KontrolerAdministratora kontrolerAdministratora = new KontrolerAdministratora(model);
		
		// Przykładowe operacje
		System.out.println("System bankomatu uruchomiony");
		System.out.println("Saldo karty: " + model.sprawdzSaldo(1));
		
		// Test wypłaty
		kontrolerKlienta.wyplataGotowki("1", "1234", 100.0);
		System.out.println("Saldo po wypłacie: " + model.sprawdzSaldo(1));
		
		// Test monitorowania
		kontrolerAdministratora.monitorowanieBezpieczenstwa();
	}
}
\end{lstlisting}

\begin{lstlisting}[caption={Kontroler/IKontrolerKlienta.java}]
package Kontroler;

public interface IKontrolerKlienta {

	public void wyplataGotowki();

	public void weryfikacjaTozsamosci();

	public void sprawdzenieStanuKonta();
}
\end{lstlisting}

\begin{lstlisting}[caption={Kontroler/KontrolerKlienta.java}]
package Kontroler;
import Model.IModel;
import java.math.BigDecimal;

public class KontrolerKlienta implements IKontrolerKlienta {
	private IModel _model;
	private WeryfikacjaTozsamosci _weryfikacja;

	public KontrolerKlienta(IModel aModel) {
		_model = aModel;
		_weryfikacja = new WeryfikacjaTozsamosci(aModel);
	}

	public void wyplataGotowki(String aNumerKarty, String aPin, double aKwota) {
		try {
			int idKarty = Integer.parseInt(aNumerKarty);
			if (_model.sprawdzPin(idKarty, aPin)) {
				BigDecimal saldo = _model.sprawdzSaldo(idKarty);
				BigDecimal kwotaWyplaty = new BigDecimal(aKwota);
				if (saldo.compareTo(kwotaWyplaty) >= 0) {
					_model.aktualizujSaldo(idKarty, kwotaWyplaty.negate());
					_model.zarejestrujZdarzenie("Wypłata gotówki: " + aKwota + " z karty: " + aNumerKarty);
				} else {
					_model.zarejestrujZdarzenie("Próba wypłaty przy niewystarczającym saldzie");
				}
			} else {
				_model.zarejestrujZdarzenie("Błędny PIN dla karty: " + aNumerKarty);
			}
		} catch (NumberFormatException e) {
			_model.zarejestrujZdarzenie("Błędny numer karty: " + aNumerKarty);
		}
	}

	public void weryfikacjaTozsamosci() {
		_weryfikacja.weryfikujPin("");
	}

	public void sprawdzenieStanuKonta() {
		_model.zarejestrujZdarzenie("Sprawdzenie stanu konta");
	}

	public void wyplataGotowki() {
		_model.zarejestrujZdarzenie("Rozpoczęcie procesu wypłaty gotówki");
	}
}
\end{lstlisting}
\begin{lstlisting}[caption={Kontroler/IKontrolerAdministratora.java}]
package Kontroler;

public interface IKontrolerAdminstratora {

	public void monitorowanieBezpieczenstwa();

	public void zdalneBlokowanieBankomatu();

	public void zarzadzanieGotowka();
}
\end{lstlisting}

\begin{lstlisting}[caption={Kontroler/WyplataGotowki.java}]
package Kontroler;
import Model.IModel;
import java.math.BigDecimal;

public abstract class WyplataGotowki {
	private IModel _model;
	private double _kwota;
	private static final double MAKSYMALNA_KWOTA_WYPLATY = 5000.0;

	public WyplataGotowki(IModel aModel) {
		_model = aModel;
		_kwota = 0.0;
	}

	public void realizujWyplate() {
		if (sprawdzanieSaldaiGotowki() && zatwierdzenieWyplaty()) {
			_model.zarejestrujZdarzenie("Zrealizowano wypłatę gotówki: " + _kwota);
		} else {
			_model.zarejestrujZdarzenie("Nie udało się zrealizować wypłaty gotówki: " + _kwota);
		}
	}

	private boolean sprawdzanieSaldaiGotowki() {
		if (_kwota <= 0) {
			return false;
		}
		if (_kwota > MAKSYMALNA_KWOTA_WYPLATY) {
			_model.zarejestrujZdarzenie("Przekroczono maksymalną kwotę wypłaty");
			return false;
		}
		return true;
	}

	private boolean zatwierdzenieWyplaty() {
		// Symulacja zatwierdzenia wypłaty
		return true;
	}

	public void ustawKwote(double aKwota) {
		_kwota = aKwota;
	}

	public double dajKwote() {
		return _kwota;
	}
}
\end{lstlisting}
\begin{lstlisting}[caption={Kontroler/KontrolerAdministratora.java}]
package Kontroler;
import Model.IModel;

public class KontrolerAdministratora implements IKontrolerAdminstratora {
	private IModel _model;
	private MonitorowanieBezpieczenstwa _monitorowanie;

	public KontrolerAdministratora(IModel aModel) {
		_model = aModel;
		_monitorowanie = new MonitorowanieBezpieczenstwa(aModel);
	}

	public void monitorowanieBezpieczenstwa() {
		_monitorowanie.rozpocznijMonitoring();
		_model.zarejestrujZdarzenie("Rozpoczęto monitoring bezpieczeństwa");
	}

	public void zdalneBlokowanieBankomatu() {
		_model.zablokujBankomat();
		_model.zarejestrujZdarzenie("Zdalne zablokowanie bankomatu przez administratora");
	}

	public void zarzadzanieGotowka() {
		_model.zarejestrujZdarzenie("Zarządzanie gotówką w bankomacie");
	}
}
\end{lstlisting}

\begin{lstlisting}[caption={Kontroler/WeryfikacjaTozsamosci.java}]
package Kontroler;
import Model.IModel;

public class WeryfikacjaTozsamosci {
	private IModel _model;
	private int _licznikProb;
	private IStrategiaZabezpieczenia _strategia;
	public IStrategiaZabezpieczenia _unnamed_IStrategiaZabezpieczenia_;
	private static final int MAKSYMALNA_LICZBA_PROB = 5;

	public WeryfikacjaTozsamosci(IModel aModel) {
		_model = aModel;
		_licznikProb = 0;
		_strategia = null;
	}

	public boolean weryfikujPin(String aPin) {
		// Weryfikacja PIN 
		_licznikProb++;
		if (_licznikProb >= MAKSYMALNA_LICZBA_PROB) {
			wykonajZabezpieczenie();
			return false;
		}
		return true;
	}

	public void ustawStrategie(IStrategiaZabezpieczenia aS) {
		_strategia = aS;
		_unnamed_IStrategiaZabezpieczenia_ = aS;
	}

	private void wykonajZabezpieczenie() {
		if (_strategia != null) {
			_strategia.wykonajReakcje(0); // ID karty - w rzeczywistości powinno być przekazane
		}
		_model.zarejestrujZdarzenie("Przekroczono maksymalną liczbę prób weryfikacji PIN");
	}

	public void resetujLicznik() {
		_licznikProb = 0;
	}
}
\end{lstlisting}

\begin{lstlisting}[caption={Kontroler/IStrategiaZabezpieczenia.java}]
package Kontroler;
import Model.IModel;

public abstract class IStrategiaZabezpieczenia {
	protected IModel _model;

	public IStrategiaZabezpieczenia(IModel aModel) {
		_model = aModel;
	}

	public void wykonajReakcje(int aIdObiektu) {
		_model.zarejestrujZdarzenie("Wykonano reakcję zabezpieczenia dla obiektu: " + aIdObiektu);
	}
}
\end{lstlisting}

\begin{lstlisting}[caption={Kontroler/ZablokowanieKarty.java}]
package Kontroler;
import Model.IModel;

public class ZablokowanieKarty extends IStrategiaZabezpieczenia {

	public ZablokowanieKarty(IModel aModel) {
		super(aModel);
	}

	public void wykonajReakcje(int aIdKarty) {
		_model.zablokujKarte(aIdKarty);
		_model.zarejestrujZdarzenie("Zablokowano kartę jako reakcję zabezpieczenia: " + aIdKarty);
	}
}
\end{lstlisting}

\begin{lstlisting}[caption={Kontroler/ZdalneBlokowanieBankomatu.java}]
package Kontroler;
import Model.IModel;

public class ZdalneBlokowanieBankomatu extends IStrategiaZabezpieczenia {

	public ZdalneBlokowanieBankomatu(IModel aModel) {
		super(aModel);
	}

	public void wykonajReakcje(int aIdBankomatu) {
		_model.zablokujBankomat();
		_model.zarejestrujZdarzenie("Zdalne zablokowanie bankomatu: " + aIdBankomatu);
	}
}
\end{lstlisting}

\begin{lstlisting}[caption={Kontroler/MonitorowanieBezpieczenstwa.java}]
package Kontroler;

import Model.IModel;


public class MonitorowanieBezpieczenstwa {
	private IModel _model;
	private boolean _monitoringAktywny;

	public MonitorowanieBezpieczenstwa(IModel aModel) {
		_model = aModel;
		_monitoringAktywny = false;
	}

	public void rozpocznijMonitoring() {
		_monitoringAktywny = true;
		_model.zarejestrujZdarzenie("Rozpoczęto monitoring bezpieczeństwa");
	}

	private boolean analizaObrazu(String aStrumien) {
		// Symulacja analizy obrazu z kamery
		if (aStrumien != null && !aStrumien.isEmpty()) {
			_model.zarejestrujZdarzenie("Analiza obrazu: " + aStrumien);
			return true;
		}
		return false;
	}

	public void zatrzymajMonitoring() {
		_monitoringAktywny = false;
		_model.zarejestrujZdarzenie("Zatrzymano monitoring bezpieczeństwa");
	}
}
\end{lstlisting}

\newpage

% --- PAKIET MODEL ---
\subsection{Pakiet Model}

\begin{lstlisting}[caption={Model/IModel.java}]
package Model;

import java.math.BigDecimal;
public interface IModel {

	public String pobierzDaneKarty(int aId);

	public boolean sprawdzPin(int aId, String aPin);

	public BigDecimal sprawdzSaldo(int aId);

	public void aktualizujSaldo(int aId, BigDecimal aKwota);

	public void zablokujKarte(int aId);

	public void zablokujBankomat();

	public void zarejestrujZdarzenie(String aOpis);

	public void usuniecieKlienta(int aNrKlienta);

	public void usuniecieKarty(int aIdKarty);
}
\end{lstlisting}

\begin{lstlisting}[caption={Model/Model.java}]
package Model;
import java.math.BigDecimal;

public class Model implements IModel {
	private Inwentarz _inwentarz;
	private IDAO _dao;
	private boolean _bankomatZablokowany;

	public Model(Inwentarz aInwentarz, IDAO aDao) {
		_inwentarz = aInwentarz;
		_dao = aDao;
		_bankomatZablokowany = false;
	}

	public void zarejestrujZdarzenie(String aOpis) {
		_dao.dodajWpisDoRejestruZdarzen(aOpis);
	}

	public String pobierzDaneKarty(int aId) {
		for (IKlient klient : _inwentarz.pobierzWszystkichKlientow()) {
			if (klient != null) {
				IKarta karta = klient.pobierzKarte(aId);
				if (karta != null) {
					return "Karta ID: " + aId + ", Saldo: " + karta.pobierzSaldo();
				}
			}
		}
		return null;
	}

	public boolean sprawdzPin(int aId, String aPin) {
		for (IKlient klient : _inwentarz.pobierzWszystkichKlientow()) {
			if (klient != null) {
				IKarta karta = klient.pobierzKarte(aId);
				if (karta != null) {
					return karta.sprawdzPin(aPin);
				}
			}
		}
		return false;
	}

	public BigDecimal sprawdzSaldo(int aId) {
		for (IKlient klient : _inwentarz.pobierzWszystkichKlientow()) {
			if (klient != null) {
				IKarta karta = klient.pobierzKarte(aId);
				if (karta != null) {
					return karta.pobierzSaldo();
				}
			}
		}
		return BigDecimal.ZERO;
	}

	public void aktualizujSaldo(int aId, BigDecimal aKwota) {
		for (IKlient klient : _inwentarz.pobierzWszystkichKlientow()) {
			if (klient != null) {
				IKarta karta = klient.pobierzKarte(aId);
				if (karta != null) {
					karta.zmienSaldo(aKwota);
					zarejestrujZdarzenie("Zmiana salda karty " + aId + " o kwotę: " + aKwota);
					return;
				}
			}
		}
	}

	public void zablokujKarte(int aId) {
		for (IKlient klient : _inwentarz.pobierzWszystkichKlientow()) {
			if (klient != null) {
				IKarta karta = klient.pobierzKarte(aId);
				if (karta != null && karta instanceof Karta) {
					((Karta) karta).ustawZablokowana(true);
					_inwentarz.zablokujKarte(aId);
					zarejestrujZdarzenie("Zablokowano kartę: " + aId);
					return;
				}
			}
		}
	}

	public void zablokujBankomat() {
		_bankomatZablokowany = true;
		zarejestrujZdarzenie("Bankomat został zablokowany");
	}

	public void usuniecieKlienta(int aNrKlienta) {
		_inwentarz.usunKlienta(aNrKlienta);
		zarejestrujZdarzenie("Usunięto klienta: " + aNrKlienta);
	}

	public void usuniecieKarty(int aIdKarty) {
		for (IKlient klient : _inwentarz.pobierzWszystkichKlientow()) {
			if (klient != null) {
				IKarta karta = klient.pobierzKarte(aIdKarty);
				if (karta != null) {
					// Usunięcie karty z listy klienta
					zarejestrujZdarzenie("Usunięto kartę: " + aIdKarty);
					return;
				}
			}
		}
	}

	public boolean czyBankomatZablokowany() {
		return _bankomatZablokowany;
	}
}
\end{lstlisting}

\begin{lstlisting}[caption={Model/IDAO.java}]
package Model;

public interface IDAO {

	public void dodajWpisDoRejestruZdarzen(String aZdarzenie);

	public String znajdzKlienta(int aNrKlienta);

	public int dodajKlienta(String aKlient);

	public void edytujKlienta(int aNrKlienta);

	public void usunKlienta(int aNrKlienta);

	public boolean zmianaBlokadyKarty(int aIdKarty);
}
\end{lstlisting}

\begin{lstlisting}[caption={Model/DAO.java}]
package Model;

import java.util.ArrayList;
import java.util.HashMap;
import java.util.List;
import java.util.Map;

public class DAO implements IDAO {
	private List<String> _rejestrZdarzen;
	private Map<Integer, String> _klienci;
	private Map<Integer, Boolean> _blokadyKart;
	private int _nastepnyNrKlienta;

	public DAO() {
		_rejestrZdarzen = new ArrayList<String>();
		_klienci = new HashMap<Integer, String>();
		_blokadyKart = new HashMap<Integer, Boolean>();
		_nastepnyNrKlienta = 1;
	}

	public void dodajWpisDoRejestruZdarzen(String aZdarzenie) {
		_rejestrZdarzen.add(aZdarzenie);
	}

	public String znajdzKlienta(int aNrKlienta) {
		return _klienci.get(aNrKlienta);
	}

	public int dodajKlienta(String aKlient) {
		int nrKlienta = _nastepnyNrKlienta++;
		_klienci.put(nrKlienta, aKlient);
		return nrKlienta;
	}

	public void edytujKlienta(int aNrKlienta) {
		// Implementacja edycji klienta
		if (_klienci.containsKey(aNrKlienta)) {
			// Logika edycji - w tym przypadku pozostawiamy bez zmian
		}
	}

	public void usunKlienta(int aNrKlienta) {
		_klienci.remove(aNrKlienta);
	}

	public boolean zmianaBlokadyKarty(int aIdKarty) {
		Boolean obecnyStan = _blokadyKart.get(aIdKarty);
		boolean nowyStan = (obecnyStan == null) ? true : !obecnyStan;
		_blokadyKart.put(aIdKarty, nowyStan);
		return nowyStan;
	}
}
\end{lstlisting}

\begin{lstlisting}[caption={Model/Inwentarz.java}]
package Model;

import java.util.ArrayList;
import java.util.List;

public class Inwentarz {
	private IDAO _dao;
	private List<IKlient> _klienci;

	public Inwentarz(IDAO aDao) {
		_dao = aDao;
		_klienci = new ArrayList<IKlient>();
	}

	public IKlient dajKlienta(int aNrKlienta) {
		for (IKlient klient : _klienci) {
			if (klient != null && klient.dajNrKlienta() == aNrKlienta) {
				//System.out.println("Znaleziono klienta: " + klient.dajNrKlienta()+ klient.dajImie()+ klient.dajNazwisko()+ klient.dajPesel());
				return klient;
			}
		}
		return null;
	}

	public void usunKlienta(int aNrKlienta) {
		IKlient klient = dajKlienta(aNrKlienta);
		if (klient != null) {
			_klienci.remove(klient);
			_dao.usunKlienta(aNrKlienta);
		}
	}

	public void zablokujKarte(int aIdKarty) {
		_dao.zmianaBlokadyKarty(aIdKarty);
		for (IKlient klient : _klienci) {
			if (klient != null) {
				IKarta karta = klient.pobierzKarte(aIdKarty);
				if (karta != null) {
					// Karta zostanie zablokowana przez model
				}
			}
		}
	}

	public List<IKlient> pobierzWszystkichKlientow() {
		return _klienci;
	}

	public void dodajKlienta(IKlient aKlient) {
		if (aKlient != null) {
			_klienci.add(aKlient);
		}
	}
}
\end{lstlisting}

\subsubsection*{Struktura Danych i Wzorzec Dekorator}

\begin{lstlisting}[caption={Model/IKlient.java}]
package Model;

public interface IKlient {

	public void dodajKarte(IKarta aK);

	public IKarta pobierzKarte(int aId);

	public int dajNrKlienta();

	public String dajImie();

	public String dajNazwisko();

	public void ustawNazwisko(String aNazwisko);

	public int dajPesel();

	public void ustawPesel(int aPesel);
}
\end{lstlisting}

\begin{lstlisting}[caption={Model/Klient.java}]
package Model;

import java.util.ArrayList;
import java.util.List;

public class Klient implements IKlient {
	private int _nrKlienta;
	private String _imie;
	private String _nazwisko;
	private int _pesel;
	private List<IKarta> _karty;

	public Klient(int aNr, String aImie) {
		_nrKlienta = aNr;
		_imie = aImie;
		_karty = new ArrayList<IKarta>();
	}

	public void dodajKarte(IKarta aK) {
		if (aK != null) {
			_karty.add(aK);
		}
	}

	public IKarta pobierzKarte(int aId) {
		for (IKarta karta : _karty) {
			if (karta != null && karta.dajId() == aId) {
				return karta;
			}
		}
		return null;
	}

	public int dajNrKlienta() {
		return _nrKlienta;
	}

	public String dajImie() {
		return _imie;
	}

	public String dajNazwisko() {
		return _nazwisko;
	}

	public void ustawNazwisko(String aNazwisko) {
		_nazwisko = aNazwisko;
	}

	public int dajPesel() {
		return _pesel;
	}

	public void ustawPesel(int aPesel) {
		_pesel = aPesel;
	}
}
\end{lstlisting}

\begin{lstlisting}[caption={Model/IKarta.java}]
package Model;
import java.math.BigDecimal;

public interface IKarta {

	public int dajId();

	public boolean sprawdzPin(String aPin);

	public BigDecimal pobierzSaldo();

	public void zmienSaldo(BigDecimal aKwota);

	public boolean czyZablokowana();
}
\end{lstlisting}

\begin{lstlisting}[caption={Model/Karta.java}]
package Model;
import java.math.BigDecimal;

public class Karta implements IKarta {
	private int _id;
	private String _pin;
	private boolean _zablokowana;
	private BigDecimal _saldo;

	public Karta(int aId, String aPin, BigDecimal aSaldo) {
		_id = aId;
		_pin = aPin;
		_saldo = aSaldo;
		_zablokowana = false;
	}

	public int dajId() {
		return _id;
	}

	public BigDecimal pobierzSaldo() {
		return _saldo;
	}

	public void zmienSaldo(BigDecimal aKwota) {
		if (!_zablokowana) {
			_saldo = _saldo.add(aKwota);
		}
	}

	public boolean czyZablokowana() {
		return _zablokowana;
	}

	public void ustawZablokowana(boolean aZablokowana) {
		_zablokowana = aZablokowana;
	}

	public boolean sprawdzPin(String aPin) {
		if (_zablokowana) {
			return false;
		}
		return _pin != null && _pin.equals(aPin);
	}
}
\end{lstlisting}

\begin{lstlisting}[caption={Model/KartaDekorator.java}]
package Model;

import java.math.BigDecimal;

public abstract class KartaDekorator implements IKarta {
	protected IKarta _karta;

	public KartaDekorator(IKarta aKarta) {
		_karta = aKarta;
	}

	public int dajId() {
		return _karta.dajId();
	}

	public boolean sprawdzPin(String aPin) {
		return _karta.sprawdzPin(aPin);
	}

	public BigDecimal pobierzSaldo() {
		return _karta.pobierzSaldo();
	}

	public void zmienSaldo(BigDecimal aKwota) {
		_karta.zmienSaldo(aKwota);
	}

	public boolean czyZablokowana() {
		return _karta.czyZablokowana();
	}
}
\end{lstlisting}

\begin{lstlisting}[caption={Model/ZablokowanaKarta.java}]
package Model;
import java.math.BigDecimal;

public class ZablokowanaKarta extends KartaDekorator {
	private String _dataBlokady;

	public ZablokowanaKarta(IKarta aKarta, String aData) {
		super(aKarta);
		_dataBlokady = aData;
	}

	public boolean czyZablokowana() {
		return true;
	}

	public int dajId() {
		return _karta.dajId();
	}

	public boolean sprawdzPin(String aPin) {
		return false; // Zablokowana karta nie może weryfikować PIN
	}

	public BigDecimal pobierzSaldo() {
		return _karta.pobierzSaldo();
	}

	public void zmienSaldo(BigDecimal aKwota) {
		// Zablokowana karta nie może zmieniać salda
	}

	public String dajDateBlokady() {
		return _dataBlokady;
	}
}
\end{lstlisting}

\subsubsection*{Wzorzec Metoda Wytwórcza}

\begin{lstlisting}[caption={Model/IFabrykaKlienta.java}]
package Model;

public interface IFabrykaKlienta {

	public IKlient stworzKontoKlienta(Formularz aDaneFormularza);
}
\end{lstlisting}

\begin{lstlisting}[caption={Model/FabrykaKlienta.java}]
package Model;

public class FabrykaKlienta implements IFabrykaKlienta {
	private static int _nastepnyNrKlienta = 1;

	public IKlient stworzKontoKlienta(Formularz aDaneFormularza) {
		if (aDaneFormularza == null) {
			return null;
		}
		int nrKlienta = _nastepnyNrKlienta++;
		IKlient klient = new Klient(nrKlienta, aDaneFormularza.dajImie());
		klient.ustawNazwisko(aDaneFormularza.dajNazwisko());
		klient.ustawPesel(aDaneFormularza.dajPesel());
		return klient;
	}
}
\end{lstlisting}

\begin{lstlisting}[caption={Model/Formularz.java}]
package Model;

public class Formularz {
	private String _imie;
	private String _nazwisko;
	private int _pesel;

	public Formularz(String aImie, String aNazwisko, int aPesel) {
		_imie = aImie;
		_nazwisko = aNazwisko;
		_pesel = aPesel;
	}

	public String dajImie() {
		return _imie;
	}

	public String dajNazwisko() {
		return _nazwisko;
	}

	public int dajPesel() {
		return _pesel;
	}
}
\end{lstlisting}

\end{document}